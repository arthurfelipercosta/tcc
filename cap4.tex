\thispagestyle{myheadings}
\newpage

\chapter{Desenvolvimento}

Este capítulo detalha a implementação técnica da solução desenvolvida, abordando a arquitetura e as tecnologias de seus três componentes principais: o backend de dados ('\textit{api.py}'), a interface de consulta web e o assistente de comunicação via WhatsApp ('\textit{n8n\_bot.py}'). Serão descritos os fluxos de trabalho que integram essas partes, demonstrando como os objetivos do projeto foram alcançados através do desenvolvimento prático.

\section{Arquitetura do Sistema}

A arquitetura do sistema é modular e distribuída, projetada para gerenciar a aquisição, o processamento, a exposição e a entrega de informações veiculares de forma eficiente em vários canais. O projeto é composto por três pilares principais: a camada de \textbf{Processamento de Dados}, a camada de \textbf{Serviços de Backend} (API RESTful) e as camadas de \textbf{Interface com o Usuário} (Web Frontend e Bot WhatsApp).

\subsection{Visão Geral da Arquitetura}
O diagrama da Figura \ref{fig:arquitetura_geral} a seguir ilustra a arquitetura geral do sistema e a interação entre seus principais componentes.
\newpage
\begin{figure}[htbp]
    \centering
    \includegraphics[width=\textwidth]{Imagens/visao_geral.png}
    \caption{Visão geral da arquitetura do sistema.}
    \label{fig:arquitetura_geral}
\end{figure}

Os principais componentes desta arquitetura são:
\begin{itemize}
    \item \textbf{Fontes de Dados Externas:} Incluem o dataset do PBEV (convertido de PDF para CSV) e a FIPE API.
    \item \textbf{Processamento de Dados:} O \textit{Jupyter Notebook} (\textit{fix.ipynb}) e a biblioteca Pandas são responsáveis pela limpeza e padronização do dataset PBEV.
    \item \textbf{Backend (API Python):} Implementado com Flask (\textit{api.py}), serve como um ponto de acesso central para os dados processados e para a integração com a FIPE API.
    \item \textbf{Frontend Web:} Uma aplicação baseada em HTML, CSS e JavaScript (\textit{index.html}, \textit{styles.css}, \textit{script.js}) que consome a API Flask para permitir a consulta e comparação visual de veículos.
    \item \textbf{Bot WhatsApp (N8N/Twilio):} Utiliza a plataforma N8N e a API do Twilio (\textit{n8n\_bot.py}) para gerenciar as interações com os usuários via WhatsApp, fazendo requisições à API Flask.
\end{itemize}

Esta estrutura garante a separação de responsabilidades e facilita a manutenção e escalabilidade do sistema.

\section{Processamento de Dados}

\subsection{Transformação de Dados PBEV (PDF para CSV)}
A base de dados PBEV, originalmente em formato PDF, foi convertida para CSV utilizando o serviço online Convertio \cite{CONVERTIO}. Este processo é crítico para viabilizar a manipulação programática dos dados. O resultado é um arquivo CSV ('\textit{dados\_corrigidos.csv}') que servirá como entrada para as próximas etapas.

\subsection{Limpeza e Otimização com Pandas no Jupyter Notebook}
O arquivo CSV do PBEV foi submetido a um rigoroso processo de limpeza e otimização. Este processo foi realizado no ambiente Jupyter Notebook (\textit{fix.ipynb}), utilizando a biblioteca Pandas \cite{Pandas}. As principais operações incluíram:
\begin{itemize}
    \item Carregamento dos dados do '\textit{dados\_corrigidos.csv}' para um DataFrame Pandas.
    \item Tratamento de valores ausentes, inconsistências e erros de formatação.
    \item Padronização de strings e valores numéricos para facilitar a busca e comparação.
    \item Remoção de linhas totalmente duplicadas, gerando um arquivo final denominado '\textit{dados\_corrigidos\_sem\_duplicatas.csv}', que é a versão otimizada da base de dados do PBEV.
\end{itemize}

O resultado final deste processo é o arquivo '\textit{dados\_corrigidos\_sem\_duplicatas.csv}', uma base de dados otimizada e pronta para ser consumida pela API.

\section{Desenvolvimento do Backend (API)}
O backend da aplicação foi desenvolvido em Python utilizando o microframework Flask \cite{FLASK}, e seu código-fonte está no arquivo \textit{n8n\_bot.py}. Este servidor atua como a fonte de dados para as interações do usuário, servindo dados e lógica de negócios para a aplicação web e para o assistente conversacional do WhatsApp.

Esta API é responsável por carregar e processar o dataset do PBEV, realizar consultas a APIs externas como a da FIPE \cite{FIPE}, e expor uma série de endpoints RESTful que atendem a diferentes necessidades:


\begin{itemize}
    \item \textbf{Para a Interface Web:} O principal endpoint consumido pelo frontend é o \textit{/api/info\_carro}. Ele fornece os dados de preço de um veículo específico, permitindo que a interface web enriqueça as informações de especificações que são carregadas localmente.

    \item \textbf{Para o Assistente Conversacional (N8N):} O backend oferece um conjunto rico de endpoints para dar suporte à lógica do bot, incluindo:
    \begin{itemize}
        \item \textbf{Validação:} Rotas como \textit{/api/validar\_marca} e \textit{/api/versoes} para guiar o usuário na seleção de um veículo.
        \item \textbf{Comparação:} A rota \textit{/api/comparativo}, que realiza uma análise detalhada entre dois veículos e retorna mensagens formatadas para o WhatsApp.
        \item \textbf{Análise:} O endpoint \textit{/api/top10}, que permite consultas estatísticas sobre o catálogo de veículos.
        \item \textbf{Contexto para IA:} A rota \textit{/api/dados\_para\_ia}, que estrutura os dados técnicos de forma otimizada para ser interpretada por modelos de linguagem.
    \end{itemize}
\end{itemize}

\section{Desenvolvimento da Interface Web}
A interface web foi desenvolvida para oferecer uma experiência de usuário rica e interativa para a consulta e comparação de veículos. As tecnologias utilizadas foram HTML para a estrutura, CSS para a estilização e JavaScript para a lógica e interatividade.

\subsection{Estrutura e Funcionalidades}
A interface web (\textit{index.html}) contém todos os elementos de interação. Toda a lógica dinâmica e a interatividade são gerenciadas pelo arquivo \textit{script.js}, e o fluxo de utilização da ferramenta se desenrola de forma contínua em uma única visualização, ilustrado pelas figuras a seguir.

A \textbf{Figura \ref{fig:interface_filtros}} exibe a área de interação principal. Nela, o usuário pode definir os critérios de busca (como Categoria e Marca) e selecionar os campos de comparação desejados. Todos esses elementos coexistem no mesmo layout, permitindo um controle constante sobre a pesquisa e visualização.

\begin{figure}[H]
    \centering
    \includegraphics[width=0.8\linewidth]{Imagens/fig_interface_filtros.png}
    \caption{Tela da interface web, destacando os filtros.}
    \label{fig:interface_filtros}
\end{figure}

Conforme os filtros são aplicados, as listas de modelos encontrados e suas respectivas versões são atualizadas dinamicamente dentro da mesma interface. A \textbf{Figura \ref{fig:interface_selecao}} detalha este processo de seleção, onde o usuário escolhe um modelo e suas opções de versão são reveladas na mesma tela, sem transições ou recarregamentos.

\begin{figure}[H]
    \centering
    \includegraphics[width=0.8\linewidth]{Imagens/fig_interface_selecao.png}
    \caption{Atualização dinâmica da interface com listas de modelos e seleção de versões.}
    \label{fig:interface_selecao}
\end{figure}

Após a seleção de dois veículos, a área de comparativo é preenchida na mesma visualização. A \textbf{Figura \ref{fig:interface_tabela}} apresenta a tabela detalhada, exibindo os atributos dos veículos lado a lado, juntamente com um sistema de pontuação visual que destaca o desempenho em cada critério.

\begin{figure}[H]
    \centering
    \includegraphics[width=0.8\linewidth]{Imagens/fig_interface_tabela.png}
    \caption{Tabela de comparação de veículos com indicadores visuais na mesma tela.}
    \label{fig:interface_tabela}
\end{figure}

Para finalizar a análise, um resumo em linguagem natural é gerado e exibido de forma integrada, como ilustrado na \textbf{Figura \ref{fig:interface_resumo}}. Este resumo, apresentado na mesma interface, traduz os dados técnicos em frases de fácil compreensão, consolidando a informação para o usuário final.

\begin{figure}[H]
    \centering
    \includegraphics[width=0.8\linewidth]{Imagens/fig_interface_resumo.png}
    \caption{Resumo comparativo gerado automaticamente, integrado à interface web.}
    \label{fig:interface_resumo}
\end{figure}

Toda essa interatividade e dinamismo são controlados pelo arquivo \textit{script.js}, que é responsável por:

\begin{itemize}
    \item Carregar os dados veiculares do arquivo \textit{dados\_corrigidos\_sem\_duplicatas.csv}.
    \item Popular os menus de seleção e listas.
    \item Gerenciar os eventos de clique do usuário.
    \item Realizar requisições ao backend para obter os preços.
    \item Construir e atualizar a tabela, a pontuação e o resumo na tela.
\end{itemize}

\section{Assistente Conversacional (WhatsApp)}
O assistente conversacional foi implementado para prover acesso rápido às informações veiculares diretamente pelo WhatsApp, utilizando uma arquitetura baseada em automação de fluxo de trabalho.

\subsection{Fluxo de Comunicação}
O fluxo de interação é orquestrado pela plataforma N8N \cite{N8N} e utiliza a API da Twilio \cite{TWILIO} como gateway de comunicação com o WhatsApp. O processo ocorre da seguinte forma:

\begin{enumerate}
    \item O usuário envia uma mensagem para o número de WhatsApp fornecido pela Twilio e associado ao serviço.
    \item A Twilio recebe a mensagem e a encaminha para um webhook configurado no N8N.
    \item O fluxo de trabalho no N8N é acionado, processando o texto da mensagem para identificar a solicitação do usuário.
    \item O N8N realiza uma requisição HTTP para a API Flask (\textit{api.py}) para buscar os dados solicitados.
    \item A API retorna os dados para o N8N, que formata a informação em uma mensagem de texto amigável.
    \item O N8N se utiliza da API da Twilio para enviar a resposta de volta ao usuário no WhatsApp.
\end{enumerate}

A centralização de toda a lógica de negócio em um único arquivo (\textit{n8n\_bot.py}) garante a consistência dos dados em todas as plataformas e simplifica a manutenção e a escalabilidade do projeto.

Este fluxo permite uma interação assíncrona e eficiente, aproveitando a robustez da API de backend e a flexibilidade da plataforma de automação. A Figura \ref{fig:n8n_geral} ilustra o fluxo de trabalho desenhado na interface do N8N.

\begin{figure}[H]
    \centering
    \includegraphics[width=0.8\linewidth]{Imagens/fig_n8n_geral.png}
    \caption{Fluxo de trabalho implementado no N8N.}
    \label{fig:n8n_geral}
\end{figure}

As \textbf{Figuras \ref{fig:whats_carro_1}}, \textbf{\ref{fig:whats_carro_2}} e \textbf{\ref{fig:whats_final}} apresentam uma sequência de interação real com o assistente de conversas no WhatsApp, demonstrando as etapas de seleção de veículos e obtenção de informações comparativas.

\begin{figure}[H]
    \centering
    \includegraphics[width=0.5\linewidth]{Imagens/fig_whats_carro_1.png}
    \caption{Interação inicial com o assistente via WhatsApp para seleção do primeiro veículo.}
    \label{fig:whats_carro_1}
\end{figure}
\begin{figure}[H]
    \centering
    \includegraphics[width=0.5\linewidth]{Imagens/fig_whats_carro_2.png}
    \caption{Continuação da conversa para seleção do segundo veículo no WhatsApp.}
    \label{fig:whats_carro_2}
\end{figure}

\begin{figure}[H]
    \centering
    \includegraphics[width=0.5\textwidth]{Imagens/fig_whats_final.png}
    \caption{Respostas do assistente no WhatsApp com o comparativo de preços e poluentes.}
    \label{fig:whats_final}
\end{figure}

\section{Considerações Finais do Capítulo}
Este capítulo detalhou as implementações técnicas e a arquitetura do sistema, cobrindo o processamento de dados, o desenvolvimento do backend unificado, e as interfaces de usuário web e bot no WhatsApp. As escolhas tecnológicas e as estratégias de desenvolvimento foram apresentadas com o objetivo de demonstrar a viabilidade e a eficácia da solução proposta para a comparação de veículos.