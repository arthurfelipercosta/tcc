\thispagestyle{myheadings}
\newpage

\chapter{Conclusão}

Este capítulo apresenta a conclusão do Trabalho de Conclusão de Curso. O trabalho teve como objetivos principais automatizar a obtenção e o tratamento de dados veiculares, desenvolver interfaces de usuário intuitivas para consulta e comparação de veículos, e prover acesso a essas informações através de um assistente conversacional no WhatsApp. Por meio do desenvolvimento de um sistema completo, foi possível integrar diversas tecnologias para alcançar essas metas, criando uma ferramenta prática e acessível para o consumidor. As principais contribuições deste projeto incluíram a consolidação e padronização de dados complexos do Programa Brasileiro de Etiquetagem Veicular (PBEV) em um formato consumível por aplicações, o desenvolvimento de uma interface web interativa que permite a comparação visual e detalhada entre veículos, e a integração com APIs externas (como a FIPE) para enriquecer a base de dados com informações de preço. Essa integração ocorreu após a tentativa e o abandono da abordagem de web scraping direto, que se mostrou inviável devido a inconsistências e bloqueios das páginas web. Adicionalmente, o projeto demonstrou a viabilidade de uma arquitetura modular para gerenciar e distribuir informações de forma eficiente em múltiplos canais.

Apesar dos resultados satisfatórios e da concretização dos objetivos propostos, este trabalho apresenta algumas limitações inerentes ao escopo e às ferramentas utilizadas. Existe uma dependência da fonte de dados PBEV, que, sendo originalmente em formato PDF, requer um processo de conversão para CSV que nem sempre é perfeito; futuras atualizações anuais demandariam a repetição manual, introduzindo um ponto de dependência. Além disso, a cobertura de veículos é limitada àqueles categorizados ou listados no PBEV, o que restringe o escopo da ferramenta e pode gerar resultados vazios para consultas de carros fora dessa base. Os desafios iniciais na aquisição de dados de preço via web scraping em plataformas como a Webmotors, devido à instabilidade e mecanismos de proteção, levaram à adoção da FIPE API como uma fonte mais robusta e confiável. A complexidade das respostas do assistente conversacional no WhatsApp, embora funcional para consultas diretas, possui limitações na interpretação de linguagem natural complexa ou perguntas ambíguas, resultando em respostas potencialmente menos flexíveis que uma interação humana. Por fim, a ausência de persistência dinâmica de dados secundários, como os preços FIPE consultados em tempo real, poderia ser otimizada com um mecanismo de cache local para reduzir a latência e a dependência contínua de APIs externas.

Com base nas limitações identificadas e nas oportunidades de aprimoramento, sugere-se a continuidade deste projeto através de diversas linhas de pesquisa e desenvolvimento, visando expandir a robustez e a inteligência da solução. Dentre elas, destacam-se a automação avançada da extração de dados PBEV, investigando o uso de técnicas de Processamento de Linguagem Natural (PLN) ou modelos de aprendizado de máquina para extrair dados de novas versões do PDF de forma automatizada, minimizando a dependência de conversões manuais. Propõe-se também a expansão da base de dados com a integração de outras APIs ou fontes para veículos não cobertos pelo PBEV e o enriquecimento com informações como histórico de manutenção, seguro ou avaliações de proprietários. O aprimoramento da inteligência conversacional do assistente do WhatsApp, integrando-o com modelos de linguagem mais avançados (LLMs) para permitir interações mais naturais, complexas e contextuais, transformando-o em um verdadeiro consultor veicular, é outra linha importante. A implementação de cache ou um banco de dados local para armazenar preços consultados frequentemente reduziria a latência e a dependência de APIs externas. O desenvolvimento de um sistema de recomendação, utilizando os dados comparativos para sugerir veículos com base nas preferências do usuário, e a otimização e escalabilidade do backend e interfaces para lidar com um volume maior de usuários e requisições, preparando a solução para um cenário de uso em larga escala, completam as sugestões para trabalhos futuros.