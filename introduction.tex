\pagestyle{myheadings}
\newpage

\chapter{Introdução}
\thispagestyle{myheadings}

\section{Contextualização}
Com o cenário de crescente digitalização dos mercados e o volume massivo de dados disponíveis online, surgem diariamente novas oportunidades e desafios. No setor automotivo, por exemplo, a busca por veículos que atendam a critérios específicos, seja por parte de compradores comuns ou de profissionais, ainda pode ser um processo demorado e ineficiente, pois a coleta, comparação e monitoramento dessas informações de forma manual demandam tempo e recursos significativos, limitando a tomada de decisão. Inicialmente, foi considerada a extração automatizada de dados diretamente de plataformas de anúncios como o Webmotors, por meio de técnicas de web scraping, mas restrições impostas pelo site inviabilizaram a continuidade dessa abordagem. Diante disso, e da necessidade de uma fonte de dados estável e confiável, este Trabalho de Conclusão de Curso (TCC) propõe o desenvolvimento de um sistema automatizado capaz de consultar e processar informações veiculares a partir de fontes oficiais, como a FIPE API criada por Deivid Fortuna. O script \textit{n8n\_bot.py} surge, portanto, como uma solução para minimizar as dificuldades inerentes à prospecção manual de informações, oferecendo um mecanismo mais eficiente para notificar o usuário sobre as oportunidades mais relevantes e otimizar o tempo de pesquisa. A principal base de dados utilizada neste projeto é um conjunto de informações em formato PDF disponibilizado pelo Programa Brasileiro de Etiquetagem Veicular (PBEV), que contém dados padronizados sobre eficiência energética e emissões, enriquecendo a análise do sistema. A integração de bibliotecas Python, como \textit{Pandas} para manipulação de dados e um framework web como \textit{Flask} ou \textit{FastAPI} para gerenciar as consultas, aliada ao poder de automação do \textit{N8N} e à comunicação via \textit{Twilio}, visa criar um assistente inteligente para busca e comparação de veículos. Além disso, o uso da API da \textit{OpenAI} permite refinar e sumarizar os dados coletados, tornando as respostas mais interpretativas e humanizadas. O projeto busca demonstrar a versatilidade e integração dessas ferramentas, explorando diferentes canais de notificação — como interface web e mensagens via WhatsApp — e avaliando sua eficácia na entrega de informações personalizadas.

\section{Objetivos}

\subsection{Objetivo Geral}
O objetivo geral deste trabalho é desenvolver um assistente automatizado para busca de informações sobre os veículos listados no PBEV, capaz de coletar, processar e entregar resultados de forma proativa e ajustável às preferências do usuário.

\subsection{Objetivos Específicos}
Para alcançar esse objetivo, foram definidos várias etapas específicas:
\begin{itemize}
    \item Criar um módulo para extração e processamento de dados veiculares em formato CSV obtidos de arquivos PDF do site do governo com dados do PBEV.
    \item Desenvolver uma API para consultar e expor esses dados.
    \item Integrar o sistema com a plataforma \textit{N8N} para orquestrar o fluxo automatizado.
    \item Configurar o envio e recebimento de mensagens via \textit{Twilio} no WhatsApp.
    \item Aplicar inteligência artificial por meio da API da \textit{OpenAI} para aprimorar a análise.
\end{itemize}

\subsection{Contribuições do trabalho}
A justificativa para este projeto está na crescente demanda por soluções que tornem a interação humana com dados estruturados mais eficiente, acessível e rápida. Ao automatizar a consulta e comparação de veículos, o \textit{n8n\_bot.py} demonstra ser uma ferramenta prática e robusta, capaz de economizar tempo, aprimorar a qualidade das informações obtidas e apoiar as decisões fundamentadas em um contexto real de inovação tecnológica.

\subsection{Estrutura da Monografia}
Este trabalho está organizado nos seguintes capítulos:
\begin{itemize}
    \item Capítulo 2: "Fundamentação Teórica", apresenta os conceitos e tecnologias onde se baseia a solução encontrada.
    \item Capítulo 3: "Metodologia", aqui é descrita as etapas e ferramentas utilizadas na construção do trabalho.
    \item Capítulo 4: "Desenvolvimento", detalha a implementação teórica do sistema.
    \item Capítulo 5: "Resultados e Análise", este capítulo discute as funcionalidades entregues e o desempenho.
    \item Capítulo 6: "Conclusão", aqui é retomado os objetivos, sintetizando um resumo das principais contribuições, discute as limitações encontradas e sugere melhorias futuras.
\end{itemize}
\vspace{1em}