linhas 28~36
original:
\begin{itemize}
    \item Ambiente Jupyter Notebook para manipulação e limpeza de dados \cite{Jupyter}
    \item Pandas e Python para processamento dos arquivos CSV \cite{Pandas,Python}
    \item Convertio para conversão de documentos \cite{CONVERTIO}
    \item Flask para desenvolvimento e exposição de API \cite{Flask}
    \item HTML, CSS e JavaScript para o desenvolvimento da interface web
    \item N8N e Twilio para integração de automação no canal WhatsApp \cite{N8N,Twilio}
    \item Dados oficiais INMETRO/PBEV e FIPE API \cite{INMETRO2024,FIPE}
\end{itemize}

sugestão:
// ... existentes
A pesquisa é classificada como aplicada, de natureza exploratória e descritiva, envolvendo o desenvolvimento de protótipo funcional. As principais ferramentas e tecnologias empregadas, todas referenciadas em detalhes na Bibliografia, incluem o ambiente Jupyter Notebook para manipulação e limpeza de dados \cite{Jupyter}, as bibliotecas Pandas em Python para processamento dos arquivos CSV \cite{Pandas,Python}, e o serviço Convertio para a conversão de documentos \cite{CONVERTIO}. O desenvolvimento da solução envolveu o framework Flask para a exposição da API \cite{Flask}, tecnologias como HTML, CSS e JavaScript para a interface web, e as plataformas N8N e Twilio para a automação no WhatsApp \cite{N8N,Twilio}. As fontes de dados primárias foram os arquivos oficiais do INMETRO/PBEV e a FIPE API \cite{INMETRO2024,FIPE}.
// ... existentes