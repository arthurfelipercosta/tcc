\thispagestyle{myheadings}
\newpage

\chapter{Conclusão}

Este capítulo apresenta a conclusão do Trabalho de Conclusão de Curso, retomando os objetivos propostos, sintetizando as principais contribuições, discutindo as limitações do trabalho e sugerindo caminhos para pesquisas e desenvolvimentos futuros.

\section{Retomada dos Objetivos e Contribuições}
Neste trabalho, os objetivos principais foram automatizar a obtenção e o tratamento de dados veiculares, desenvolver interfaces de usuário intuitivas para consulta e comparação de veículos, e prover acesso a essas informações através de um assistente conversacional no WhatsApp. Através do desenvolvimento de um sistema completo, foi possível integrar diversas tecnologias para tentar alcançar essas metas, criando uma ferramenta prática e acessível para o consumidor.

As principais contribuições deste projeto incluem:
\begin{itemize}
    \item A consolidação e padronização de dados complexos do Programa Brasileiro de Etiquetagem Veicular (PBEV) em um formato consumível por aplicações.
    \item O desenvolvimento de uma interface web interativa que permite a comparação visual e detalhada entre veículos.
    \item A integração com APIs externas (como a FIPE) para enriquecer a base de dados com informações de preço, após a tentativa e abandono da abordagem de web scraping direto devido a inconsistências e bloqueios das páginas web à esse procedimento de aquisição de dados.
    \item A demonstração da viabilidade de uma arquitetura modular para gerenciar e distribuir informações de forma eficiente em múltiplos canais.
\end{itemize}

\section{Limitações do Trabalho}
Apesar dos resultados satisfatórios e da concretização de alguns dos objetivos propostos, este trabalho apresenta algumas limitações inerentes ao escopo e às ferramentas utilizadas:
\begin{itemize}
    \item \textbf{Dependência da Fonte de Dados PBEV:} A base de dados principal, sendo originalmente disponibilizada em formato PDF, requer um processo de conversão para CSV, que muitas vezes não é perfeito. Embora o serviço do Convertio tenha sido eficaz, futuras atualizações anuais do PBEV demandariam a repetição desse processo manual, o que introduz um ponto de dependência e esforço recorrente.
    \item \textbf{Cobertura de Veículos Fora do PBEV:} O sistema é fundamentalmente baseado nos dados do PBEV. Veículos que não são categorizados ou listados neste programa não poderão ser processados ou comparados, o que limita o escopo da ferramenta a uma parte específica do mercado automotivo e pode gerar erros ou resultados vazios para consultas de carros fora dessa cobertura.
    \item \textbf{Desafios na Aquisição de Dados (Web Scraping):} A tentativa inicial de aquisição de dados de preço via web scraping em plataformas como a Webmotors apresentou desafios significativos, principalmente devido à instabilidade e à dificuldade de manter a integridade dos dados face a possíveis alterações na estrutura das páginas. Esta limitação inerente ao web scraping levou à decisão de procurar outros métodos de aquisição de dados, o que resultou em encontrar e integrar a FIPE API ao projeto, o que ofereceu uma fonte mais robusta e confiável para os valores de mercado.
    \item \textbf{Complexidade das Respostas do Assistente Conversacional:} A inteligência do assistente do WhatsApp, embora funcional para consultas diretas e comparações estruturadas, possui limitações na interpretação de linguagem natural complexa ou perguntas ambíguas, resultando em respostas potencialmente menos flexíveis do que uma interação humana.
    \item \textbf{Ausência de Persistência Dinâmica de Dados Secundários:} Informações como os preços FIPE são consultadas em tempo real. Não há um mecanismo de persistência local para esses dados, o que poderia otimizar o desempenho e reduzir a dependência de APIs externas.
\end{itemize}


\section{Trabalhos Futuros}
Com base nas limitações identificadas e nas oportunidades de aprimoramento, sugere-se a continuidade deste projeto através das seguintes linhas de pesquisa e desenvolvimento, visando expandir a robustez e a inteligência da solução:
\begin{itemize}
    \item \textbf{Automação Avançada da Extração de Dados PBEV:} Investigar o uso de técnicas de Processamento de Linguagem Natural (PLN) ou modelos de aprendizado de máquina para extrair dados de novas versões do PDF do PBEV de forma automatizada, minimizando a dependência de conversões manuais.
    \item \textbf{Expansão da Base de Dados e Fontes Adicionais:}
        \begin{enumerate}
            \item Integrar o sistema com outras APIs ou fontes de dados para veículos não cobertos pelo PBEV.
            \item Enriquecer com informações como histórico de manutenção, seguro, ou avaliações de proprietários.
        \end{enumerate}
    \item \textbf{Aprimoramento da Inteligência Conversacional:} Integrar o assistente do WhatsApp com modelos de linguagem mais avançados (LLMs) para permitir interações mais naturais, complexas e contextuais, transformando-o em um consultor veicular mais completo e eficaz.
    \item \textbf{Implementação de Cache ou Banco de Dados Local para Preços:} Desenvolver um mecanismo de cache ou um pequeno banco de dados local para armazenar preços consultados frequentemente, reduzindo a latência e a dependência contínua de APIs externas.
    \item \textbf{Desenvolvimento de um Sistema de Recomendação:} Utilizar os dados comparativos para desenvolver um sistema que sugira veículos com base nas preferências e prioridades do usuário (ex: "carro mais econômico para família", "veículo com menor emissão").
    \item \textbf{Otimização e Escalabilidade:} Realizar testes de desempenho e otimização para garantir que o backend e as interfaces possam lidar com um volume maior de usuários e requisições, preparando a solução para um cenário de uso em larga escala.
\end{itemize}