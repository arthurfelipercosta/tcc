% margens
\usepackage[top=3cm, left=3cm, right=2cm, bottom=2cm]{geometry}

\usepackage{setspace}
\setstretch{1.5}

\usepackage[footnotesize]{caption}

\usepackage{verbatim}

% espacamento entre parágrafos
\setlength{\parindent}{1.25cm}

% colocar parágrafo no começo das seções
\usepackage{indentfirst}

% aceitar acentos, cedilhas, tils
\usepackage[utf8]{inputenc}
\usepackage[T1]{fontenc}
\usepackage[brazilian]{babel}

% simbolos matemáticos
\usepackage{amsmath}
\usepackage{amsthm}
\usepackage{amsfonts}
\usepackage{amssymb}

% para subitem
\usepackage{enumitem}

% para sub sub item
\usepackage{outlines}

% para incluir imagens
\usepackage{graphicx}

% para incluir URLs
\usepackage[hyphens]{url}
\usepackage{hyperref}

% arrumar tabelas e imagens nos seus lugares
\usepackage{float}

% caminho das imagens
\graphicspath{{Imagens/}}


% insere paginas em pdf (usar para inserir a ficha) 
\usepackage{pdfpages}

% retira primeira página do capítulo
\usepackage{etoolbox}
\patchcmd{\chapter}{plain}{myheadings}{}{}
\patchcmd{\part}{plain}{empty}{}{}

% para criar teoremas
\newtheorem{thm}{Teorema}[section]
\newtheorem{exe}{Exemplo}[section]
\newtheorem{dfn}{Definição}[section]
\newtheorem{prob}{Problema}[section]
\newtheorem{cor}{Corolário}[section]
\newtheorem{prop}{Proposição}[section]
\newtheorem{lem}{Lema} [section]
\newcounter{contar}

% tira a palavra capitulo dos capitulos (ABNT)
\usepackage{titlesec}
\titleformat{\thechapter}{\huge\bf}{\thechapter}{30pt}{\huge\bf}

% ==========================
% FORMATAÇÃO DO SUMÁRIO (TOC)
% ==========================
\usepackage{tocloft}
% Remove o recuo (alinhar todos à esquerda)
\setlength{\cftchapindent}{0pt}
\setlength{\cftsecindent}{0pt}
\setlength{\cftsubsecindent}{0pt}
\setlength{\cftsubsubsecindent}{0pt}
% Ajusta espaçamento dos pontinhos
\renewcommand{\cftdotsep}{1}

% -------- ESTILOS --------
% CAPÍTULO (ex: 1 FUNDAMENTAÇÃO TEÓRICA) — negrito e MAIÚSCULO
\renewcommand{\cftchapfont}{\bfseries}
\renewcommand{\cftchappagefont}{\bfseries}

% Força maiúsculas nos capítulos
\makeatletter
\let\oldl@chapter\l@chapter
\renewcommand*{\l@chapter}[2]{%
  \oldl@chapter{\MakeUppercase{#1}}{#2}%
}
\makeatother

% SEÇÃO (ex: 1.1) — negrito e minúsculo
\renewcommand{\cftsecfont}{\bfseries}
\renewcommand{\cftsecpagefont}{\bfseries}
% SUBSEÇÃO (ex: 1.1.1) — normal
\renewcommand{\cftsubsecfont}{\normalfont}
\renewcommand{\cftsubsecpagefont}{\normalfont}
% SUBSUBSEÇÃO (ex: 1.1.1.1) — normal e itálico
\renewcommand{\cftsubsubsecfont}{\itshape}
\renewcommand{\cftsubsubsecpagefont}{\itshape}

% Ajustes para a bibliografia
\makeatletter
\renewcommand\@biblabel[1]{}
\makeatother
\setlength{\bibident}{0pt}
\newcommand{\citecustom}[1]{(\textsc{#1})}